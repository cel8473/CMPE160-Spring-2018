\documentclass[CMPE]{KGCOEReport}

\usepackage{float}
\newcommand{\classCode}{CMPE 160}  
\newcommand{\name}{Christopher Larson}
\newcommand{\LabSectionNum}{L3}
\newcommand{\LabInstructor}{Mr.\ Dominguez}	
\newcommand{\TAs}{TA Andrew Ramsey \\ TA Matthew Miller \\ TA Madeline Mooney}
\newcommand{\LectureSectionNum}{01}
\newcommand{\LectureInstructor}{Professor Beato}
\newcommand{\exerciseNumber}{04}
\newcommand{\exerciseDescription}{Combinational Logic Circuit Design Using Boolean Algrebra Simplification }
\newcommand{\dateDone}{7 February 2018}
\newcommand{\dateSubmitted}{14 February 2018}

\begin{document}
\maketitle

\section*{Abstract}
The objective of this exercise is to simplify Boolean algebraic expressions to a simplified expression and implement the expressions using a combinational logic circuit. Combination logic circuits provide control signals to device or perform mathematical operations. The logic gate designed in the exercise read a a single two-bit binary number, N= (N1,N0), and produced a single four-bit output, F=(W,X,Y,Z), which depends on the selection control signal, "C". If \textit{C} was "0" then the output value would be the square of N,  (F = $N^2$), and if \textit{C} was "1" then the output value would be fives times N, (F= 5N). The control signal, two-bit input and the four-bit output were recorded in a table and used to create the Boolean expressions used for the circuit. The physical cicuit performed the correct function that was theorized on the truth table.
\section*{Design Methodology}
The exercise used a two-bit binary number, N=(N1,N0) and a control signal to determine the output of the four-bit binary number shown in Table 1. The control signal determined whether the output was found using F= $N^2$ when \textit{C} was "0" and F = 5N when \textit{C} was "1".
\begin{table}[H]
	\centering
	\caption{Truth table for mathematical operations $N^2$ and 5N}
	\label{tab:Table 1}
	\begin{tabular}{|ccc||cccc|}
		\hline
		C & N1 & N0 & W & X & Y & Z\\ \hline
		0 & 0 & 0 & 0 & 0 & 0 & 0\\ \hline
		0 & 0 & 1 & 0 & 0 & 0 & 1\\ \hline
		0 & 1 & 0 & 0 & 1 & 0 & 0\\ \hline
		0 & 1 & 0 & 1 & 0 & 0 & 1\\ \hline
		1 & 0 & 0 & 0 & 0 & 0 & 0\\ \hline
		1 & 0 & 1 & 0 & 1 & 0 & 1\\ \hline
		1 & 1 & 0 & 1 & 0 & 1 & 0\\ \hline
		1 & 1 & 1 & 1 & 1 & 1 & 1\\
		\hline
	\end{tabular}
\end{table}

The outputs were then used to create sum of products Boolean expression, where each product term included all of the input variables. Using Boolean algebra properties the Boolean expressions were simplified so they could be implemented with the minimal parts from the lab kit. The simplification and Boolean properties are shown in Table 2.

\section*{Results and Analysis}

\section*{Conclusion}

\end{document}