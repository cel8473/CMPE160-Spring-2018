\documentclass[11pt]{article}
\usepackage[margin-1in]{geometry}
\usepackage{graphicx}
\usepackage{amsmath}
\begin{document}
\begin{table}[II]
	\centering
	\caption{Put the descriptive table caption here}
	\label{tab:MyAwesomeTableLabel}
	\begin{tabular}{lcr}
		Col 1 & Col 2 & Col 3 \\ \hline
		Hello & This is a cell & Tableee\\
		Wheee & Very Important Information & Yes\\
		Row 1 & Row 2 & Row 3\\
	\end{tabular}
\end{table}

\begin{figure}[II]
	\centering
	\includegraphics[width=0.35\textwidth]{somecatCrookshanks}
		->extension necessary
	\caption{This is some cat, Crookshanks}
	\label{fig:myAwesomeImageLAbel}
\end{figure}

\begin{equation}
	\overline{A + B} - \overline{A}\,\overline{B}
	\label{eq:SuperEQ}
\end{equation}

\begin{enumerate}
	\item The 1st item
	\item The 2nd item
	\item
		\begin{enumerate}
			\item Nested is useful
			\item More nesting?
		\end{enumerate}
\end{enumerate}

\begin{table}[II]
	\cetnering
	\caption{A Table with Multi Columns}
	\label{tab:MultiColTable}
	\begin{tabular{lcr}
		\multilcolumn{2}{c}{Col 1 anD 2} & \\ \cline{1-2}
		Col 1 & Col 2 & Col 3 \\ \hline
		Hello & This is a cell & Tableee\\
		Wheee & Very Important Information & Yes\\
		Row 1 & Row 2 & Row 3\\
	\end{tabular}
\end{table}

\begin{figure}[H]
	\centering
	\begin{Karnaugh}
	\content{0,1,0,0,1,0,1,1,0,1,0,0,0,0,1,1}
	\implicantLateral[2pt]{4}{6}{green}
	\implicant{7}{14}{green}
	\implicantTopBottom{1}{9}{red}
	\end{Karnaugh}
	\caption{A K-Map}
	\label{fig:MyKMAp|
\end{figure}

\newpage
\LARGE
\medskip		

\end{document}