 \documentclass[CMPE]{KGCOEReport}

\usepackage{float}
\usepackage{graphicx}
\usepackage{amsmath}
\usepackage{kmap}
\newcommand{\classCode}{CMPE 160}  
\newcommand{\name}{Christopher Larson}
\newcommand{\LabSectionNum}{L3}
\newcommand{\LabInstructor}{Mr.\ Dominguez}	
\newcommand{\TAs}{Andrew Ramsey \\ Matthew Millar \\ Madeline Mooney}
\newcommand{\LectureSectionNum}{01}
\newcommand{\LectureInstructor}{Professor Beato}
\newcommand{\exerciseNumber}{06}
\newcommand{\exerciseDescription}{Sequential Circuit Elements }
\newcommand{\dateDone}{28 February 2018}
\newcommand{\dateSubmitted}{6 March 2018}

\begin{document}
\maketitle

\section*{Abstract}
The objective of this exercise was to design, and implement an active-low enable D latch and a rising edge-triggered, master-slave D flip-flop. An active-low D latch is used to control the output while the enable is low and to leave the output as the last input when the enable is high. A D flip-flop is made by combinining two D latches together and using a controlled CLK signal that changes on a timed interval. A rising edge-triggered flip-flop has two states, one between the rising edges of the clock wave where D changes and one where D remains the same across two rising edges of the clock. A schematic for both the D latch and D flip-flop were designed and tested to see if the design was correct by comparing the waveforms to Table 1 for the D latch and Table 2 for the rising edge-triggered D flip-flop.

\begin{table}[H]
	\centering
	\caption{Active-low enable D latch}
	\label{tab: Table 1}
	\begin{tabular}{|c|c||c|c|}
		\hline
		En & D & Q & Qn\\ \hline
		0 & 0 & 0 & 1\\ \hline
		0 & 1 & 1 & 0\\ \hline
		1 & X & Q & Qn\\ \hline
		\hline
	\end{tabular}
\end{table}

Table 1 is a D latch with an active-low enable.

\begin{table}[H]
	\centering
	\caption{Rising edge-triggered D flip-flop}
	\label{tab: Table 1}
	\begin{tabular}{|c|c||c|c|}
		\hline
		CLK & D & Q & Qn\\ \hline
		$\uparrow$ & 0 & 1 & 0\\ \hline
		$\uparrow$ & 1 & 0 & 1\\ \hline
		Otherwise & X & Q & Qn\\ \hline
		\hline
	\end{tabular}
\end{table}

\end{document}